\documentclass[letterpaper]{report}

\title{Predicting Sea Levels Using the Actuaries Climate Index}
\date{April 7th, 2017}
\author{Jennifer Mince, Karl Maier, Wesley Merrick, Katie Vervack, Toby Duncan}
\begin{document}
	\maketitle

		Our group would like to use the Actuaries Climate Index data set to answer the central question: how do the seasonal components of high temperatures, low temperatures, heavy rain, and drought affect the seasonal sea levels in North American regions and by how much?
		The research we want to conduct is both important and relevant to the societal need of knowing what causes sea levels to rise and predicting future measurements to be better prepared for future problems. Higher sea levels have a significant impact on both local and global environments. Rising levels can cause destructive storms to move closer inland and as a result, there is more coastal flooding. \textquotedblleft€œIn the United States, almost 40 percent of the population lives in relatively high-population-density coastal areas, where sea level plays a role in flooding, shoreline erosion, and hazards from storms" (NOAA). Rising seas threaten the infrastructure used by industries and businesses, such as roads, bridges, power plants, and oil wells. The coastline of the United States is heavily populated and the risk of rising sea levels causes many problems. \textquotedblleft Approximately 25 million people live in an area vulnerable to coastal flooding"€ (EPA). The U.S. economy, transportation, and many ecosystems are threatened by the impending flooding that comes with the rising levels. Integral coastal activities including marine transportation of goods, resource extraction, and tourism help to generate \textquotedblleft€œ58\% of the national gross domestic product" (EPA). Transportation in the United States is substantially affected by coastal flooding, especially roads. In low-lying communities, the streets are first to be flooded because they are lower than the surrounding land (Titus). The current drainage systems for these roads are not efficient enough to handle increased frequency of flooding due to rising sea levels. 
		Using this data, our group could help this societal need by introducing a model for the rate at which the sea levels are increasing, and calculating factors that contribute to the increase in levels. The data set has an Actuary Climate Index (ACI) created with the components: frequency of temperatures above the 90th percentile (T90), frequency of temperatures below the 10th percentile (T10), maximum rainfall per month in five consecutive days (Rx5), annual maximum consecutive dry days (CDD), frequency of wind speed above the 90th percentile (W), and sea level changes (SL). It is an objective measure of changes in extreme weather and changes in sea level relative to the base period of 1961 through 1990. This index is an educational tool designed to help inform actuaries, public policymakers, and the general public on changes in these measures over recent decades (Actuary Climate Index). Our group will create our own ACI using the seasonal values for T90, T10, Rx5, and CDD to see how they affect the sea level changes. 
		The Actuaries Climate Index data set needed to be cleaned up in order for our team to effectively and efficiently for our model. In order to clean up the data, we first needed to understand which variables we wanted to keep as well as determine what each of the variables meant. Since the Central Arctic (CAR) and Midwest (MID) regions had no sea level data so we excluded it from the final clean data set. We are trying to predict how sea level is affected and it would be impossible to predict with out the data for sea level. We had also excluded the monthly variable sheets since our group decided we only wanted to do the seasonal components. Now that we had all our data we wanted we needed to add it all into one sheet on excel. Having all the data on one sheet in excel would allow the code to easily import one csv file rather than multiple csv files. Once all the data was on one sheet in excel, we had to format it. We noticed that columns initially were the years and the rows were the actual variables. We thought it would be more effective to switch the rows and columns so that the years were the rows and the variables were on the columns. Excel has a function built in called Transpose, which allowed us to easy flip the rows and columns. Lastly, we had to clean up the variable names. For example, we changed the initial variable of ACI\_sealevel\_seasonal\_ALA.csv to SL ALA. We then had a clean easy to read and understand data set that would be implanted into our code. 

		
	
\end{document}
